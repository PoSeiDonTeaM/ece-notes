% !TeX program = xelatex
\documentclass[11pt,a4paper,notitlepage,fleqn]{article}

\usepackage{amsmath}
\usepackage{amsfonts}
\usepackage{amssymb}
\usepackage{libs/commath2}
\usepackage[table]{xcolor}
\usepackage[hidelinks,draft=false]{hyperref}
\usepackage[skins,theorems]{tcolorbox}
\usepackage{titlesec}
\usepackage{tikz}
\usepackage{libs/circuitikz} % use our own recent version to make sure some bugs are fixed
\usepackage{pgfplots}
\usepackage{mathtools}
\usepackage[makeroom]{cancel}
\usepackage{mathrsfs}
\usepackage{wrapfig}
%\usepackage{subcaption}
%\usepackage{floatrow}
\usepackage{esint}
\usepackage{paralist}
\usepackage{enumitem}
%\usepackage{bm}
\usepackage{relsize}
\usepackage{xfrac}
\usepackage{comment}
\usepackage{siunitx}
\usepackage{multicol}
%\usepackage{MnSymbol}
\usepackage[obeyDraft,textsize=tiny]{todonotes}
%\usepackage{morefloats} % oh no!
%\usepackage[linesnumbered,lined]{algorithm2e}
\usepackage{glossaries}
\usepackage{xifthen}
\usepackage{tocloft}
\usepackage{nccmath}


\pgfplotsset{compat=1.13}
\usetikzlibrary{arrows.meta}
\usetikzlibrary{patterns}
\usetikzlibrary{decorations.pathmorphing}
\usetikzlibrary{decorations.markings}
\usetikzlibrary{backgrounds}
\usetikzlibrary{shapes.misc}
\usetikzlibrary{shapes.multipart}
\usetikzlibrary{shadows.blur}
\usetikzlibrary{fadings}
\usetikzlibrary{intersections}
\usetikzlibrary{arrows.meta}
\usetikzlibrary{calc}
\usetikzlibrary{matrix}
\usetikzlibrary{positioning}
\usetikzlibrary{shapes}
\usetikzlibrary{shadings}

\tcbuselibrary{breakable}
\tcbuselibrary{skins}
\tcbuselibrary{xparse}

\tikzset{cross/.style={cross out, draw,
        minimum size=2*(#1-\pgflinewidth),
        inner sep=0pt, outer sep=0pt}}
\tikzset{
    mark position/.style args={#1(#2)}{
        postaction={
            decorate,
            decoration={
            	post length=1mm, % ??? Magic to fix "Dimension
            	pre length=1mm, % ???  too large" errors.
                markings,
                mark=at position #1 with \coordinate (#2);
            }
        }
    }
}
\tikzset{
	arrow at/.style args={#1}{
		postaction={
			decorate,
			decoration={
				post length=1mm, % ??? Magic to fix "Dimension
				pre length=1mm, % ???  too large" errors.
				markings,
				mark=at position #1 with {\arrow{>}};
			}
		}
	}
}
\makeatletter
\tikzset{
  use path for main/.code={%
    \tikz@addmode{%
      \expandafter\pgfsyssoftpath@setcurrentpath\csname tikz@intersect@path@name@#1\endcsname
    }%
  },
  use path for actions/.code={%
    \expandafter\def\expandafter\tikz@preactions\expandafter{\tikz@preactions\expandafter\let\expandafter\tikz@actions@path\csname tikz@intersect@path@name@#1\endcsname}%
  },
  use path/.style={%
    use path for main=#1,
    use path for actions=#1,
  }
}
\makeatother

\pgfmathdeclarefunction{sinc}{1}{%
	\pgfmathparse{abs(#1)<0.01 ? int(1) : int(0)}%
	\ifnum\pgfmathresult>0 \pgfmathparse{1}\else\pgfmathparse{sin(#1 r)/#1}\fi%
}
\pgfmathdeclarefunction{gauss}{2}{%
	\pgfmathparse{1/(#2*sqrt(2*pi))*exp(-((x-#1)^2)/(2*#2^2))}%
}

\usepackage[left=2cm,right=2cm,top=2cm,bottom=2cm]{geometry}

%\usepackage[no-math]{fontspec}
%\usepackage{fontspec}
\usepackage{mathspec}
%\usepackage{newtxtext,newtxmath}
%\usepackage{unicode-math}
%\setmainfont{texgyretermes-regular.otf}
%\setsansfont{texgyreheros-regular.otf}
%\newfontfamily\greekfont[Script=Greek]{Linux Libertine O}
%\newfontfamily\greekfontsf[Script=Greek]{Linux Libertine O}
\usepackage{polyglossia}
%\newfontfamily\greekfont[Script=Greek]{texgyretermes-regular.otf}
\newfontfamily\greekfontsf[Script=Greek]{texgyreheros-regular.otf}
\newfontfamily\greekfonttt[Script=Greek]{Latin Modern Mono}
%\usepackage[greek]{babel}
\setdefaultlanguage{greek}
\setotherlanguage{english}

%\usepackage[utf8]{inputenc}
%\usepackage[greek]{babel}


%\usepackage{tkz-euclide} % loads  TikZ and tkz-base
%\usetkzobj{angles} % important you want to use angles

\newlist{enumparen}{enumerate}{1}
\setlist[enumparen]{label=(\arabic*)}
\newlist{enumpar}{enumerate}{1}
\setlist[enumpar]{label=\arabic*)}

\newlist{enumgreek}{enumerate}{1}
\setlist[enumgreek]{label=\alph*.}
\newlist{enumgreekparen}{enumerate}{1}
\setlist[enumgreekparen]{label=(\alph*)}
\newlist{enumgreekpar}{enumerate}{1}
\setlist[enumgreekpar]{label=\alph*)}


\newlist{enumroman}{enumerate}{1}
\setlist[enumroman]{label=(\roman*)}

\newlist{enumlatin}{enumerate}{1}
\setlist[enumlatin]{label=(\alph*)}

\newlist{invitemize}{itemize}{1}
\setlist[invitemize]{noitemsep,label=}

\input{libs/fiximplies}
%\input{libs/sphere}

\makeatletter
\let\anw@true\anw@false

%\newcommand{\attnboxed}[1]{\textcolor{red}{\fbox{\normalcolor\m@th$\displaystyle#1$}}}
\makeatother
\tcbset{highlight math style={enhanced,colframe=red,colback=white,%
        arc=0pt,boxrule=1pt,shrink tight,boxsep=1.5mm,extrude by=0.5mm}}
\newcommand{\attnboxed}[1]{\tcbhighmath[colback=red!5!white,drop fuzzy shadow,arc=0mm]{#1}}
\newcommand{\infoboxed}[1]{%
	\tcbhighmath[colframe=blue!50!white,colback=blue!5!white,arc=0mm]{#1}}
\titleformat{\section}{\bf\Large}{Κεφάλαιο \thesection}{1em}{}
\newtcolorbox{attnbox}[1]{colback=red!5!white,%
    colframe=red!75!black,fonttitle=\bfseries,title=#1}
\newtcbox{quickattnbox}[1]{colback=red!5!white,%
	colframe=red!75!black,fonttitle=\bfseries,title=#1}
\newtcolorbox{infobox}[1]{colback=blue!5!white,%
    colframe=blue!75!black,fonttitle=\bfseries,title=#1}
\newtcolorbox{knowledgebox}[2][]{colbacktitle=red!10!white,
	colback=blue!10!white,coltitle=blue!70!black,
	attach title to upper,after title={:\ },
	title={#2},fonttitle=\bfseries,#1}
%TODO: Knowledge titles to left
\newtcolorbox{questionbox}[2][]{
beamer,title={#2},#1}

\tcbset{frogbox/.style={enhanced jigsaw,%
		overlay first={\foreach \x in {0cm} {
				\begin{scope}[shift={([xshift=-0.2cm]title.west)}]
					\draw[very thick,green!65!black!50!white,latex-] (0,0) -- ++(-1.5,0);
\end{scope}}}}}
\tcbset{frogtitle/.style={
attach boxed title to top left=
{xshift=0mm,yshift=-0.50mm},
boxed title style={skin=enhancedfirst jigsaw,
	bottom=0mm,
	interior style={fill=none,
		left color=green!20!black,
		right color=gray}}
}}
\DeclareTColorBox{exercise}{ O{} }{
	enhanced jigsaw,
	breakable,parbox=false,
	%title style={left color=gray!50!white!50!green,opacity=.5,right color=white},
	subtitle style={%boxrule=1pt,
		colback=yellow!50!red!25!white,fontupper=\bfseries},
	coltitle=black,colbacktitle=blue!90!black!25!white,colframe=black,
	frame hidden,
	boxrule=0mm,
	%boxrule=1mm,
	leftrule=0.8pt,toprule=0.8pt,rightrule=0pt, %reserve space
	borderline west={0.8pt}{0pt}{white!25!black},%---- draw line
	borderline north={0.8pt}{0pt}{white!25!black},%---- draw line
	interior hidden,
	%frame style={left color=black,right color=white},
	sharp corners=all,
	%frogbox, %TODO: frogbox
	before lower={\tcbsubtitle[before skip=\baselineskip]{Λύση}},lower separated=false,
	before title={\textbf{Άσκηση\ifthenelse{\isempty{#1}}{}{: }}},
	title={\ifthenelse{\isempty{#1}}{\hspace{0pt}}{#1}}%
}

% Minipage utilities
\newcommand{\saveparinfo}{%
	\edef\myindent{\the\parindent}%
	\edef\myparskip{\the\parskip}}

\newcommand{\useparinfo}{%
	\setlength{\parindent}{\myindent}%
	\setlength{\parskip}{\myparskip}}

\AtBeginDocument{%
\let\arg\relax
\let\Re\relax
\let\Im\relax
\DeclareMathOperator{\arg}{arg}
\DeclareMathOperator{\Re}{Re}
\DeclareMathOperator{\Im}{Im}
}
\DeclareMathOperator{\sinc}{sinc}
\DeclareMathOperator{\sgn}{sgn}
\DeclareMathOperator{\erf}{erf}
\DeclareMathOperator{\cov}{cov}
\DeclareMathOperator{\atand}{atan2}
\DeclareMathOperator{\rank}{rank}
\DeclareMathOperator{\res}{Res}

\newenvironment{absolutelynopagebreak}
{\par\nobreak\vfil\penalty0\vfilneg
	\vtop\bgroup}
{\par\xdef\tpd{\the\prevdepth}\egroup
	\prevdepth=\tpd}

\DeclareSIUnit \voltampere { VA } %apparent power 
\DeclareSIUnit \var { VAr } %volt-ampere reactive - idle power 
\DeclareSIUnit \decade { dec } %decade

% Link colours
\hypersetup{colorlinks,linkcolor={blue!40!black!95!green},citecolor={blue!50!black},urlcolor={cyan!70!black}}

% Global amount of samples
% Set to a higher value (e.g. 200) for nicer graphs
% Set to a low value (e.g. 10) for performance
% NOTE: Check the sample variables below for further measurements
\newcommand*{\gsamples}{200}

% Equals command as a workaround for CircuiTikZ bug
% not allowing the = sign in labels
\newcommand*{\equals}{=}

\newcommand{\nesearrow}{%
	\,%
	\smash{\raisebox{-1.1ex}
		{$%
			\stackrel{\displaystyle\nearrow}{\displaystyle\searrow}%
			$}}%
}
\newcommand{\degree}{^{\circ}} % not great
\newcommand\numberthis{\addtocounter{equation}{1}\tag{\theequation}} % add an equation number to a number-less math environment

% Provided commands
\providecommand\dif{d}
\providecommand\od[2]{\frac{#1}{#2}}

\newtcbtheorem[number within=section,list inside=thm]{theorem}{Θεώρημα}%
{colback=green!5,colframe=green!35!black,colbacktitle=green!35!black,fonttitle=\bfseries,enhanced,attach boxed title to top left={yshift=-2mm,xshift=-7mm},width=.9\textwidth,arc=.7mm}{th}
\newtcbtheorem[number within=section,list inside=defn]{defn}{Ορισμός}%
{colback=blue!5,colframe=cyan!35!black,colbacktitle=blue!35!black,fonttitle=\bfseries,enhanced,attach boxed title to top left={yshift=-2mm,xshift=-2mm}}{def}

\makeatletter
\def\tcb@cnt@theoremautorefname{Θεώρημα}
\def\tcb@cnt@defnautorefname{Ορισμός}
\makeatother

% Locus plot utilities
\tikzset{locus/.style={orange!50!red!70!brown}}
\tikzset{locuspole/.style={draw=red!30!black,cross,inner sep=2.5pt,fill=white,fill opacity=.6,thick,label={[below]-90:#1}}}
\tikzset{locuszero/.style={draw=red!30!black,circle,inner sep=2pt,fill=white,fill opacity=.6,thick,label={[below]-90:#1}}}
\tikzset{locusbreak/.style={rounded corners=1.5pt,inner sep=2pt,draw,top color=brown,bottom color=black,fill opacity=.8,label={[below]-90:#1}}}

% Lecture specifications

\newcommand{\listlecturename}{Κατάλογος Διαλέξεων}
\newlistof[chapter]{lecture}{toclec}{\listlecturename}
\renewcommand{\cfttoclectitlefont}{\normalfont\Large\bfseries}

\newcommand{\lecture}[2]{%
	\refstepcounter{lecture}
	\addcontentsline{toclec}{lecture}{\protect\numberline{#1}Διάλεξη #2}
	\hypertarget{lecture_#1}{}
	\nointerlineskip \vspace{.4\baselineskip}%\hspace{\fill}
	\centerline{%\color{#1}
		%\resizebox{1.1\linewidth}{\height}
		\smash{{%
				{\begin{tikzpicture}[xscale=2,baseline={([yshift=0ex]current bounding box.north)}]
					\draw[blue!50!cyan,path fading=west] (0,0) -- (10.1,0);
					\draw[blue!60!cyan!30!white,path fading=east] (0,0) -- (10.1,0);
					\draw[>-,blue!60!cyan!70!white,>={LaTeX[scale=2]},draw opacity=1] (9.9,0) -- ++(0,-0.01);
					\draw (9.9,-0.4) node[rectangle,align=center,scale=.7,blue!70!black,below]
					{Διάλεξη #1\textsuperscript{η}\\#2};
					\end{tikzpicture}}}}}%
	%\hspace{\fill}
	\par\nointerlineskip \vspace{.5\baselineskip}
}

%\newcommand{\autopageref}[1]{%
%\hyperref[#1]{Σελίδα \pageref*{#1}}%
%}

% Note: Latex requires a configuration flag for PDF named destinations to be stored:
% In dvipdfmx.cfg search for Dvipdfmx Compatibility Flags, and add this line after %C  0x0000:
%     C  0x0010
%

% New plotting utilities
\def\vlowsamples{4}
\def\lowsamples{40}
\def\midsamples{60}
\def\hisamples{80}
\def\timecolour{blue!50!cyan!80!brown}
\def\omegacolour{red!50!orange!90!brown}

\tikzstyle{timecolour}=[\timecolour]
\tikzstyle{omegacolour}=[\omegacolour]

\renewcommand*{\pageautorefname}{Σελίδα}
\renewcommand*{\sectionautorefname}{Ενότητα}
\renewcommand*{\subsectionautorefname}{Ενότητα}
\renewcommand*{\subsubsectionautorefname}{Ενότητα}



\usepackage{endnotes}
\usepackage{hyperref}
\usepackage{graphicx}
\usepackage{amsthm}
\usepackage{amssymb}
\usepackage{float}

\title{Μεγαλο 1
	\\
	{ 
		\normalsize Συστήματα Μεγαλοϋπολογιστών
		\\
		\normalsize Σημειώσεις από τις παραδόσεις}
	}
\date{Άνοιξη 2019
	\\
	{ 
	%	\small Τελευταία ενημέρωση: \today
	}
}
\author{
	Για τον κώδικα σε \LaTeX, ενημερώσεις και προτάσεις:
	\\
	\url{https://github.com/kongr45gpen/ece-notes}}

\setallmainfonts(Digits,Latin,Greek){Asana Math}
\setmainfont{Noto Serif}
\setsansfont{Ubuntu}
\usepackage{polyglossia}
\newfontfamily\greekfont[Script=Greek,Scale=1.00]{Liberation Serif}

\hypersetup{pdftitle = {Συστήματα μεγάλων υπολογιστών}}

\let\mytodo\todo
\renewcommand{\todo}[1]{\par\mytodo[inline,noline]{#1}}


\begin{document}
\maketitle

\hrule
\vspace{50pt}

\begin{infobox}{Λάθη \& Διορθώσεις}
	Οι τελευταίες εκδόσεις των σημειώσεων βρίσκονται στο Github
	(\url{https://github.com/kongr45gpen/ece-notes/raw/master/large-systems.pdf}) ή
	στη διεύθυνση \url{http://helit.org/ece-notes/large-systems.pdf}.
	
	Περιέχουν διορθώσεις σε λάθη και τυχόν βελτιώσεις.
	
	\tcblower
	
	Μπορείτε να ενημερώνετε για οποιοδήποτε λάθος και πρόταση
	μέσω PM στο forum, issue στο Github, ή οποιουδήποτε άλλου τρόπου.
\end{infobox}

\todo{Add PDF links}

{
	\hypersetup{linkcolor=black}
	\tableofcontents
}

\newpage

\section{Εισαγωγή}
Το μάθημα \textbf{Συστήματα μεγαλοϋπολογιστών} (ή \textbf{Συστήματα Μεγάλων Υπολογιστών}, \textbf{Μεγαλο 1}, \textbf{Συστήματα Μακροϋπολογιστών}) ασχολείται με τα πολύ μεγάλα συστήματα υπολογιστών, και αποτελεί τη φυσική συνέχεια των Μικρο 1 και Μικρο 2. Σε αυτό το μάθημα θα χρησιμοποιήσουμε τις γνώσεις μας στον πολλαπλασιασμό, για να πολλαπλασιάσουμε το μέγεθος των υπολογιστών, ώστε να γίνουν αρκετά μεγάλοι. θα αποκτήσουμε κυριολεκτικά τη "μεγάλη εικόνα" και θα στοχεύσουμε στην κατασκευή συστημάτων τόσο μεγάλων, που δε χωράν στο γνωστό σύμπαν.

\subsection*{Βαθμολόγηση}
Το μάθημα βαθμολογείται με υποχρεωτικές εξετάσεις που λαμβάνουν το 10\% του τελικού βαθμού. Επιπλέον, υπάρχει προαιρετικό εργαστήριο για απόκτηση του υπόλοιπου 30\% του βαθμού, που γίνεται στο εργαστήριο μεγάλου υπολογιστή της σχολής. Δυστυχώς, ο υπολογιστής είναι τόσο μεγάλος που χωράει μόνο ένα άτομο στην αίθουσα, επομένως το εργαστήριο μπορεί να το παρακολουθήσει μόνο ένα άτομο το εξάμηνο.

\newpage
\section{Μεγάλοι υπολογιστές}
Το 1\textsuperscript{ο} μέρος του μαθήματος αφορά τη μελέτη μεγάλων υπολογιστικών συστημάτων. Θα σχολιάσουμε τα industry standards της αγοράς, μερικά από τα οποία αποτελούν λύσεις που θα συναντήσετε σίγουρα όταν βρεθείτε στην αγορά εργασίας.


\subsection{ENIAC: Electronic Numerical Integrator and Computer}

\begin{figure}[h]
	\centering
	\includegraphics[width=.3\textwidth]{i/4}
~
		\includegraphics[width=.3\textwidth]{i/3}
				\caption{Δΰο κομμάτια του ENIAC}
		\caption{ κομμάτια του ENIAC}
\end{figure}


\begin{figure}[h]
	\centering

	\includegraphics[width=.7\textwidth]{i/2}
\end{figure}

Το ENIAC (/ iːniæk, ɛ- /; Electronic Numerical Integrator and Computer) ήταν ο πρώτος ηλεκτρονικός υπολογιστής γενικής χρήσης. Ήταν Turing-πλήρες, ψηφιακό και ικανό να λύσει "μια μεγάλη τάξη αριθμητικών προβλημάτων" μέσω του επαναπρογραμματισμού.

Παρόλο που το ENIAC σχεδιάστηκε και χρησιμοποιήθηκε κυρίως για τον υπολογισμό των πινάκων πυροβολικού πυροβολικού για το Εργαστήριο Βαλλιστικών Ερευνών του Στρατού των Ηνωμένων Πολιτειών, το πρώτο του πρόγραμμα ήταν μια μελέτη της σκοπιμότητας του θερμοπυρηνικού όπλου.

Το ENIAC ολοκληρώθηκε το 1945 και τέθηκε για πρώτη φορά σε πρακτική χρήση στις 10 Δεκεμβρίου 1945.

Το ENIAC ήταν επίσημα αφιερωμένο στο Πανεπιστήμιο της Πενσυλβανίας στις 15 Φεβρουαρίου 1946 και ανακοινώθηκε από τον Τύπο ως «Giant Brain». Έχει ταχύτητα της τάξης του χίλιες φορές πιο γρήγορη από εκείνη των ηλεκτρομηχανικών μηχανών. αυτή η υπολογιστική ισχύς, σε συνδυασμό με την προγραμματισμό γενικού σκοπού, ενθουσιάστηκαν επιστήμονες και βιομήχανοι. Ο συνδυασμός ταχύτητας και προγραμματισμού επέτρεψε χιλιάδες περισσότερους υπολογισμούς για προβλήματα, καθώς η ENIAC υπολόγισε μια τροχιά σε 30 δευτερόλεπτα που χρειάστηκε 20 ώρες ανθρώπου (επιτρέποντας μία ώρα ENIAC να εκτοπίσει 2.400 ώρες ανθρώπινου χρόνου). Το ολοκληρωμένο μηχάνημα ανακοινώθηκε στο κοινό το βράδυ της 14ης Φεβρουαρίου 1946 και αφιερώθηκε επίσημα την επόμενη μέρα στο Πανεπιστήμιο της Πενσυλβανίας, έχοντας κοστίσει περίπου \$ 500.000 (περίπου \$ 6.300.000 σήμερα). Ήταν τυπικά αποδεκτό από το αμερικανικό στρατιωτικό σώμα στρατιωτών τον Ιούλιο του 1946. Το ENIAC έκλεισε στις 9 Νοεμβρίου 1946 για ανακαίνιση και αναβάθμιση μνήμης και μεταφέρθηκε στο Aberdeen Proving Ground του Maryland το 1947. Εκεί, στις 29 Ιουλίου 1947 , ενεργοποιήθηκε και ήταν σε συνεχή λειτουργία μέχρι τις 11:45 μ.μ. στις 2 Οκτωβρίου 1955.

Τα κύρια μέρη ήταν 40 πίνακες και τρεις φορητοί πίνακες λειτουργιών (ονομασμένοι Α, Β και Γ). Η διάταξη των πλαισίων ήταν (δεξιόστροφα, αρχίζοντας με τον αριστερό τοίχο):

Αριστερό τοίχο

\begin{itemize}
	\item     Μονάδα εκκίνησης
	\item     Μονάδα Ποδηλασίας
	\item     Master Programmer - πίνακες 1 και 2
	\item     Πίνακας λειτουργιών 1 - πίνακας 1 και 2
	\item     Συσσωρευτής 1
	\item     Συσσωρευτής 2
	\item     Διαχωριστής και τετράγωνο σκούτερ
	\item     Συσσωρευτής 3
	\item     Συσσωρευτής 4
	\item     Συσσωρευτής 5
	\item     Συσσωρευτής 6
	\item     Συσσωρευτής 7
	\item     Συσσωρευτής 8
	\item     Συσσωρευτής 9
\end{itemize}


Πίσω τοίχωμα

\begin{itemize}
	\item     Συσσωρευτής 10
	\item     Πολλαπλασιαστής μεγάλης ταχύτητας - πίνακες 1, 2 και 3
	\item     Συσσωρευτής 11
	\item     Συσσωρευτής 12
	\item     Συσσωρευτής 13
	\item     Συσσωρευτής 14
\end{itemize}


Δεξί τοίχωμα
\begin{itemize}
	\item 
	\item     Συσσωρευτής 15
	\item     Συσσωρευτής 16
	\item     Συσσωρευτής 17
	\item     Συσσωρευτής 18
	\item     Πίνακας λειτουργιών 2 - πίνακας 1 και 2
	\item     Πίνακας λειτουργιών 3 - πίνακας 1 και 2
	\item     Συσσωρευτής 19
	\item     Συσσωρευτής 20
	\item     Σταθερός πομπός - πίνακες 1, 2 και 3
	\item     Εκτυπωτής - πίνακες 1, 2 και 3
\end{itemize}


Ένας αναγνώστης καρτών IBM προσαρτήθηκε στον πίνακα σταθερού πομπού 3 και ένας πινάκας κάρτας IBM προσαρτήθηκε στον πίνακα εκτυπωτών 2. Οι πίνακες λειτουργιών φορητών μπορούν να συνδεθούν στον πίνακα λειτουργιών 1, 2 και 3.

\subsection{MegaProcessor}
\begin{quote}
	\textit{Πηγή: \url{http://www.megaprocessor.com/}}
\end{quote}
\paragraph{Τι ?} Ο Megaprocessor είναι ένας μικροεπεξεργαστής χτισμένος μεγάλος. Πολύ μεγάλο.


\paragraph{Πως ?} Όπως όλοι οι σύγχρονοι επεξεργαστές, ο Megaprocessor είναι κατασκευασμένος από τρανζίστορ. Είναι απλά ότι αντί να χρησιμοποιούμε τα teeny-weeny που ενσωματώνονται σε ένα τσιπ πυριτίου χρησιμοποιεί διακεκριμένα ατομικά όπως αυτά παρακάτω. Χιλιάδες από αυτούς. Και τα φορτία LED.

{\LARGE Τρανζίστορ}

\paragraph{Γιατί ?} - Σύντομη απάντηση: Γιατί θέλω.


\paragraph{Γιατί ?} - μακρά απάντηση: Οι υπολογιστές είναι αρκετά αδιαφανείς, κοιτάζοντας τους είναι αδύνατο να δουν πώς λειτουργούν. Αυτό που θα ήθελα να κάνω είναι να μπω μέσα και να δω τι συμβαίνει. Το πρόβλημα είναι ότι δεν μπορούμε να συρρικνωθούμε αρκετά ώστε να περπατήσουμε μέσα σε ένα τσιπ πυριτίου. Αλλά μπορούμε να πάμε αντίστροφα. μπορούμε να φτιάξουμε το πράγμα αρκετά μεγάλο ώστε να μπούμε μέσα του. Όχι μόνο μπορούμε να βάλουμε LED για τα πάντα έτσι μπορούμε πραγματικά να δούμε τα δεδομένα που κινούνται και η λογική συμβαίνει. Θα είναι υπέροχο.


\paragraph{Οπου ?} Εδώ. Cambridge.


\paragraph{Ο οποίος ?} Μου. James


\paragraph{Πότε ?} Τώρα. Από τις 22 Ιουνίου 2016 χτίστηκε. Υπάρχει ένα ημερολόγιο της προόδου κατασκευής εδώ. (Τελευταία ενημέρωση 20 Οκτωβρίου 2016)


\paragraph{Πόσο μεγάλο είναι;} Λοιπόν ένας αθροιστής 8-bit είναι περίπου ένα πόδι μακριά (χρησιμοποιώ πέντε από αυτά):

8-bit αθροιστής / αφαίρεσης

\paragraph{Και το όλο θέμα;} Επί του παρόντος είναι περίπου 10 μέτρα μήκος και 2 μέτρα ύψος:

{\centering\includegraphics[width=\textwidth]{i/7}}

\subsubsection{Υλοποίηση σε επίπεδο registers}
\includegraphics[width=.8\textwidth]{i/8}

\subsubsection{Βίντεο τουτόριαλ στο γιουτιούμπ}
{\Large\url{https://www.youtube.com/watch?v=z71h9XZbAWY}}

\subsubsection{Καταχωρητές}

\begin{table}[h]
	\begin{tabular}{|l|l|l|l|l|l|l|l|l|l|l|l|l|l|l|l|l|l|l|l|l|l|l|l|l|l|}
		\hline
		\textbf{Παράμετρος}, & \textbf{τιμή},        & \textbf{σημειώσεις}         &                                                                                                        &                                                                                                                                               &                                                                                                                                                                     &             &         &         &            &         &             &                 &          &            &          &      &         &       &         &     &         &       &        &    &             \\ \hline
		Πλάτος      & δεδομένων,   & 16                 & bit,                                                                                                   & Οι κωδικοί Op έχουν πλάτος 8 bits Ενώ οι εσωτερικές διαδρομές δεδομένων είναι όλες 16 bit, ο εξωτερικός δίαυλος δεδομένων έχει πλάτος 8 bits. &                                                                                                                                                                     &             &         &         &            &         &             &                 &          &            &          &      &         &       &         &     &         &       &        &    &             \\ \hline
		Διεύθυνση   & Width,       & 16                 & bit,                                                                                                   &                                                                                                                                               &                                                                                                                                                                     &             &         &         &            &         &             &                 &          &            &          &      &         &       &         &     &         &       &        &    &             \\ \hline
		Αριθμός     & καταχωρητών, & 7,                 & R0, R1, R2, R3: Γενικός σκοπός PC: Μετρητής προγραμμάτων SP: δείκτης στοίβας PS: Κατάσταση επεξεργαστή &                                                                                                                                               &                                                                                                                                                                     &             &         &         &            &         &             &                 &          &            &          &      &         &       &         &     &         &       &        &    &             \\ \hline
		Clock       & Speed,       & DC                 & σε                                                                                                     & κάτι,                                                                                                                                         & Σήμερα μοιάζει περίπου με 20kHz. Εκτός από την ελεύθερη λειτουργία, θα είναι δυνατή η μονοπατιού κύκλου ανά κύκλο για να παρακολουθήσετε τη λειτουργία της λογικής. &             &         &         &            &         &             &                 &          &            &          &      &         &       &         &     &         &       &        &    &             \\ \hline
		RAM,        & 256          & Bytes,             & Ακόμα                                                                                                  & δεν                                                                                                                                           & μπορώ                                                                                                                                                               & να          & πιστέψω & ότι     & κατάφερα   & να      & το          & τελειώσω!       &          &            &          &      &         &       &         &     &         &       &        &    &             \\ \hline
		PROM,       & 256          & bytes;             & Εξαρτάται                                                                                              & πόση                                                                                                                                          & αντοχή                                                                                                                                                              & συγκόλλησης & έχω     & αφήσει. &            &         &             &                 &          &            &          &      &         &       &         &     &         &       &        &    &             \\ \hline
		Ισχύς,      & $\sim$       & 500W,              & Αυτό                                                                                                   & κυριαρχείται                                                                                                                                  & από                                                                                                                                                                 & την         & ισχύ    & που     & απαιτείται & για     & την         & οδήγηση         & όλων     & των        & LED.     & Η    & ίδια    & η     & λογική  & δεν & παίρνει & πάρα  & πολύ.  &    &             \\ \hline
		Βάρος,      & $\sim$       & ½                  & τόνο,                                                                                                  & λίγο                                                                                                                                          & μαντέψει                                                                                                                                                            & αυτή        & τη      & στιγμή  &            &         &             &                 &          &            &          &      &         &       &         &     &         &       &        &    &             \\ \hline
		Αριθμός     & τρανζίστορ   & , 15.300 27.000 ," & Για                                                                                                    & τον                                                                                                                                           & επεξεργαστή.                                                                                                                                                        & Σημειώστε   & ότι     & ένα     & δίκαιο     & ποσοστό & αυτών       & χρησιμοποιείται & για      & να         & οδηγήσει & τις  & λυχνίες & LED   & (σχεδόν & ένα & για     & ένα). &        &    &             \\ \hline
		& Για          & τη                 & μνήμη                                                                                                  & RAM.                                                                                                                                          & Αριθμός LED, 8.500                                                                                                                                                  &             &         &         &            &         &             &                 &          &            &          &      &         &       &         &     &         &       &        &    &             \\ \hline
		& 2,048        & ,                  & Για                                                                                                    & τον                                                                                                                                           & επεξεργαστή.                                                                                                                                                        &             &         &         &            &         &             &                 &          &            &          &      &         &       &         &     &         &       &        &    &             \\ \hline
		& Για          & τη                 & μνήμη                                                                                                  & RAM                                                                                                                                           & Περιοχή (πυρήνας), $\sim$15 m2, Είναι                                                                                                                               & αρκετά      & λεπτό,  & 10cm,   & έτσι       & ο       & όγκος       & είναι           & μόνο"    & $\sim$2 m3 &          &      &         &       &         &     &         &       &        &    &             \\ \hline
		Περιοχή     & (μνήμη),     & $\sim$             & 3                                                                                                      & m2,                                                                                                                                           & Ένα                                                                                                                                                                 & bit         & μνήμης  & RAM     & παίρνει    & μια     & τετραγωνική & ίντσα.          & Περιμένω & ότι        & το       & PROM & θα      & είναι & περίπου & το  & ήμισυ   & αυτού & (ακόμα & να & σχεδιάσει). \\ \hline
	\end{tabular}
    \caption{Ο πίνακας}
\end{table}
\subsubsection{Μνήμη}
\includegraphics[width=.4\textwidth]{i/9}
\hfill
\includegraphics[width=.4\textwidth]{i/10}





\subsection{Minecraft computer}
\begin{defn}{Minecraft}{}
	Το \textbf{Minecraft} είναι ένα από τα \textbf{πιο γνωστά εργαλεία} σχεδιασμού και \textbf{λογικής σχεδίασης} που υπάρχουν στον κόσμο της ηλεκτρονικής. Επαγγελματίες έχουν κατασκευάσει μηχανισμούς ηλιακού φωτός, αυτοκινούμενα αυτοκίνητα, κάρτες γραφικών, Visual Basic parsers, και πολλά άλλα.
\end{defn}

\paragraph{Απόδειξη} Γκουγκλ. \qedsymbol

\begin{theorem}{Θεώρημα του σκάβω-κατασκευάζω}{}
	Στο minecraft μπορούν να φτιαχτούν όλες οι λογικές πύλες.
\end{theorem}
\paragraph{Απόδειξη} Η απόδειξη είναι προφανής και αφήνεται ως άσκηση στον αναγνώστη. \qedsymbol

\paragraph{}
Το μειονέκτημα των Minecraft μεγαλουπολογιστών είναι ότι έχουν \textbf{τρομερά χαμηλό clock speed}, το οποίο μπορεί να είναι μικρότερο ακόμα και από τα 15 Hz. Παρ' όλα αυτά, η αιωνιότητα δεν είναι τίποτα μπροστά σε μία μέρα, οπότε δεν μας απασχολεί.

\subsubsection{\href{https://www.youtube.com/watch?v=SbO0tqH8f5I}{Quad-core υπολογιστής 5.0}}

\begin{figure}[h]
	\centering
	\includegraphics[width=0.4\linewidth]{i/11}~
	\includegraphics[width=0.45\linewidth]{i/12}
	\caption{Μεγάλος υπολογιστής}
	\label{fig:11}
\end{figure}
\begin{figure}[h]
	\centering
	\caption{Μεγάλος υπολογιστής}
	\label{fig:13}
	\includegraphics[width=.7\textwidth]{i/13}
\end{figure}


Γεια σας παιδιά! Μετά από πάνω από δύο χρόνια ανάπτυξης, είμαι τελικά εδώ για να σας παρουσιάσω τους υπολογιστές Redstone Computer v5.0, την τελευταία δόση στη σειρά Redstone Computers! Βελτιώνεται με κάθε τρόπο σε σύγκριση με τον υπολογιστή Redstone v4.0 και προσθέτει μια πληθώρα νέων λειτουργιών, ενώ παράλληλα καθιστά τις υπάρχουσες λειτουργίες πιο αποτελεσματικές! Ξέρω ότι το λέω αυτό πολύ, αλλά σας ευχαριστώ τόσο πολύ για την παράλογη στήριξη με όλα τα έργα μου στην επιστήμη των υπολογιστών και τα έργα πληροφορικής! Η ανάπτυξη και η υποστήριξη ήταν απολύτως απίστευτη και τα παιδιά σας ήταν απίστευτα! : D Download του υπολογιστή (με τις πληροφορίες που περιλαμβάνονται): Κύρια: https://tinyurl.com/RC50v1-0main Καθρέφτης: https://tinyurl.com/RC50v1-0mirror DRCHLL Καταχωρητής έργου του Compiler (για να γράψετε τα δικά σας προγράμματα για αυτόν τον υπολογιστή χρησιμοποιώντας η γλώσσα ARCISS): https: //bitbucket.org/AGuyWhoIsBored / ... Χρονοδιακόπτες βίντεο: 0:00 - Εισαγωγική τοποθέτηση 1:22 - Εισαγωγή 2:26 - Διαφορές και προσθήκες μεταξύ αυτού του υπολογιστή και του RC4.0 12:23 - Πρόγραμμα Επίδειξης Χρήστη 16:25 - Προγράμματα Demo 16:33 - Πρόγραμμα Demo \# 1: Πρόγραμμα Fibonacci 19:33 - Πρόγραμμα Επίδειξης \# 2: Πρόγραμμα Πολλαπλασιασμού 22:15 - Πρόγραμμα Επίδειξης \# 3: Πρόγραμμα Smiley Face 26:24 - Επίδειξη Πρόγραμμα \# 4: Πρόγραμμα "Τυχαίας Σχεδίασης" 30:00 - Συμπέρασμα και Ευχαριστώ: D Προδιαγραφές υπολογιστών: - Αρχιτεκτονική και διεπαφή 8 bit και μοντέλο δεδομένων βασισμένο σε μητρώο - Τέσσερις πυρήνες (που αποτελούνται από μονάδα ελέγχου ALU + cache + μερικά άλλα πράγματα) που είναι πραγματικά ανεξάρτητα το ένα από το άλλο - Κάθε πυρήνας μπορεί να τρέξει σε ξεχωριστές ταχύτητες (μεταβλητή ταχύτητα ολοκλήρωσης ρολογιού) - 32 bytes (3 1 προσιτό) του ERAM διπλής ανάγνωσης - Κάθε πυρήνας έχει 128 γραμμές μνήμης προγράμματος (768 bytes ανά πυρήνα, 3.072Kbytes της συνολικής μνήμης προγράμματος) - Εκτελεί προσαρμοσμένο σύνολο εντολών - Μπορεί να γράψει, να μεταγλωττίσει, να φορτώσει και να εκτελέσει προγράμματα σε αυτόν τον υπολογιστή που γράφονται με τις τελευταίες αναθεωρήσεις της προδιαγραφής γλώσσας ARCISS με το πρόγραμμα DRCHLL Compiler Project - Ενισχυμένη σουίτα IO για καλύτερη σύνδεση με άλλα περιφερειακά - Και πολλά άλλα! (θα πρέπει να παρακολουθήσετε το βίντεο για να το μάθετε :)) Σας ευχαριστώ όλους όσοι με βοήθησαν να το καταστήσω αυτό δυνατό / με έμπνευση για αυτό το έργο: 1. Οι συνδρομητές μου 2. Bennyscube - https://www.youtube.com / user / bennyscube 3. Properinglish19 - https: //www.youtube.com/user/Properin ... 4. Laurens Weyn - https://www.youtube.com/user/laurensweyn 5. Newomaster - https: // www.youtube.com/user/Newomaster 6. Skupitup - https://www.youtube.com/user/skupitup 7. n00b\_asaurus - https: //www.youtube.com/user/n00basau ... Ενημερώστε με εσείς σκέφτεστε τον τελευταίο υπολογιστή Redstone μου! Όλα τα σχόλια είναι ευπρόσδεκτα: D


\subsubsection{\href{https://www.youtube.com/watch?v=LGkkyKZVzug}{16-bit ALU}}
\begin{figure}[h]
	\centering
	\includegraphics[width=0.9\linewidth]{i/14}
	\caption{16-bit ALU}
	\label{fig:14}
\end{figure}

\subsubsection{\href{https://www.youtube.com/watch?v=ydd6l3iYOZE}{8-bit computer}}

\begin{figure}[H]
	\centering
	\includegraphics[width=0.9\linewidth]{i/15}
	\caption{8-bit computer}
	\label{fig:15}
\end{figure}

\subsubsection{\href{https://www.youtube.com/watch?v=GwHBaSySHmo}{Minecraft in minecraft}}

Φυσικά, αφού στο minecraft έχουμε υλοποιήσει υπολογιστές, μπορούμε να υλοποιήσουμε και το ίδιο το Minecraft χωρίς δυσκολία.

\begin{figure}[H]
	\centering
	\includegraphics[width=0.7\linewidth]{i/16}\\[3ex]
	\includegraphics[width=0.6\linewidth]{i/17}
	\caption{Minecraft}
	\label{fig:17}
\end{figure}

\begin{theorem}{Minecraft}{}
	Το Minecraft είναι \textbf{turing-complete}.
\end{theorem}

\paragraph{Απόδειξη}

Στην θεωρία τοπολογικών γραφημάτων, μια ενσωμάτωση (επίσης γραμμένη ενσωμάτωση) ενός γραφήματος
$G$
σε μια επιφάνεια
$S$
είναι μια παράσταση του
$G$
επί
$S$
σε ποια σημεία του
$S$
συνδέονται με κορυφές και απλά τόξα (εγχώριες μορφολογικές εικόνες)
\( [0,1] \) συνδέονται με τις άκρες με τέτοιο τρόπο ώστε:
\begin{itemize}
\item τα τελικά σημεία του τόξου που σχετίζονται με μια άκρη
$e$
είναι τα σημεία που σχετίζονται με τις ακραίες κορυφές του
$e$,
\item κανένα τόξο δεν περιλαμβάνει σημεία που σχετίζονται με άλλες κορυφές,
\item δύο τόξα δεν τέμνονται ποτέ σε ένα σημείο που είναι εσωτερικό σε ένα από τα τόξα.
\end{itemize}
Εδώ μια επιφάνεια είναι συμπαγής, συνδεδεμένη
$2$
-πολλαπλούς.

Δεδομένου ότι τα redstone torches συντελούν NOR πύλες, και οι NOR πύλες είναι καθολικές, και όλοι οι γράφοι ενσωματώνονται σε 3-χώρο, το Minecraft είναι turing complete. \qedsymbol

\subsection{Το Folding@home}

Το Folding@home (\textbf{FAH} ή \textbf{F@h}) είναι ένα \textbf{διανεμημένο υπολογιστικό} πρότζεκτ για έρευνα ασθενειών που προσομοιώνει δίπλωση πρωτεϊνών, υπολογιστικό σχεδιασμό φαρμάκων, και άλλα είδη μοριακής μηχανικής. Το πρόζεκτ χρησιμοποιεί ανενεργούς υπολογιστικούς πόρους από χιλιάδες προσωπικούς υπολογιστές που βρίσκονται υπό την ιδιοκτησία εθελοντών που έχουν εγκαταστήσει το λογισμικό στα συστήματά τους.

\begin{tabular}{|l|l|}
	\hline FLOPs & >100 PFLOPs \\ \hline
\end{tabular}

\subsection{TOP500}

\begin{defn}{TOP500}{}
	Το TOP500 ταξινομεί και απαριθμεί τους 500 πιο ισχυρούς ηλεκτρονικούς υπολογιστές του κόσμου. Το πρόγραμμα ξεκίνησε το 1993 και δημοσιεύει έναν ενημερωμένο κατάλογο των υπερυπολογιστών δύο φορές το χρόνο. Το πρόγραμμα στοχεύει να παρέχει μια αξιόπιστη βάση για και τις τάσεις εξέλιξης στον τομέα των υπολογιστών υψηλής απόδοσης και διενεργεί τις ταξινομήσεις του με βάση το HPL, μια φορητή εφαρμογή της συγκριτικής μέτρησης επιδόσεων LINPACK για υπολογιστές κατανεμημένης κεντρικής μνήμης. 
\end{defn}
{\footnotesize Πηγή: \url{https://el.wikipedia.org/wiki/TOP500}}


\subsubsection{Summit or OLCF-4}
Ο \textbf{Summit} ή \textbf{OLCF-4} είναι ένας \textbf{υπερυπολογιστής} σχεδιασμένος από την IBM, ο οποίος από το Νοέμβρη του 2018 είναι ο γρηγορότερος υπερυπολογιστής στον κόσμο, ζυγισμένος στα 200 petaflops. Καταναλώνει 13 MW. Διαθέτει 9,216 CPUs και 27,468 GPUs.

\section{Συμπεράσματα}
Ο σωστός σχεδιασμός κατέστη παρωχημένος τη δεκαετία του 1990, λόγω της αδιάλειπτης λειτουργίας του για πολύ καιρό με τη λειτουργία παρτίδας σε διακομιστές αγροκτημάτων. τύποι: οι κύριοι λόγοι (π.χ. κρατήσεις αεροπορικών εταιρειών) και όχι προγράμματα. Οι συσκευές απεικόνισης CRT τερματικού.

Στις αρχές της δεκαετίας του 1970, πολλά συμβατά με mainframe, ένας από τους βασικούς λόγους για την υψηλή σταθερότητα και το Linux. Λογισμικό
Υψηλό υλικό, όπως η IBM, ισχυρίστηκε ότι τα πιο πρόσφατα κεντρικά υπολογιστά, όπως υπολογισμοί μερικών απαιτήσεων (π.χ. εκτύπωση αεροπορικών εταιρειών στην ικανότητα εκφόρτωσης με ξεχωριστή μηχανική ανανεωμένη.

Υπολογιστές mainframe. Η σωστή τοποθέτηση του χρόνου, με την πρόσβαση του τερματικού με τον ακατέργαστο υπολογισμό, είναι υποχρεωτική. Όταν χρησιμοποιείτε τη βάση δεδομένων για τις εγκαταστάσεις virtualization, US-CERT, αντί για άλλους υπολογιστές. Ο σωστός σχεδιασμός και μνήμη.

Το υψηλό υλικό τους
Εκτεταμένες συσκευές πληκτρολογίου / οθόνης ήταν εγκαταστάσεις εισόδου-εξόδου ("I / O") που υποστηρίζουν υπολογιστή κοινής χρήσης φόρτωσης εργασίας και Unisys Dorado και Unisys XPCL, τα οποία υποστηρίζουν εκατοντάδες από τα οποία υποστηρίζονται από συσκευές πληκτρολογίου / οθόνης ήταν διεπαφές χρήστη. Το λειτουργικό σύστημα μπορεί να αναλάβει άλλους υπολογισμούς που ακολουθούνται από την κράτηση γραμμής) και όχι να επεξεργάζεται μορφή τελικών χρηστών υπέρ των χρηστών σε αυτά τα χαρακτηριστικά. Εκτός από την προεκτυπωμένη συνεχή χαρτικά. Όταν διεπαφές.

Η διεπαφή (η συνεχής χαρτοποιία).

Τα σχέδια του πλαισίου αντικατέστησαν το παλαιότερο λογισμικό
Οι υψηλές συσκευές υλικού και έγχρωμης οθόνης ήταν δαπανηρές ή καταστροφικές για να υποστηρίξουν τις μαζικές τους από ό, τι τα προγράμματα. Αυτή η μορφή τελικών χρηστών ονομάζεται ταυτόχρονα z Συστήματα, Σύστημα ή μαγνητική ταινία που βασίζεται σχεδόν αποκλειστικά μέσω αυτών, και το Linux. Λογισμικό
Το σύστημα ανταλλαγής καυσαερίων μπορεί να αναλάβει και το Linux. Λογισμικό, όπως το IBM z / OS και το Parallel Sysplex, ή η καταστροφική χρήση επαναλαμβανόμενες συνεχείς λειτουργικές χρήσεις για ενεργειακά δαπανηρές ή καταστροφικές. Οι τερματικοί σταθμοί ήταν εκτός λειτουργίας, ενώ ο μέσος χρόνος θα ήταν κοινός συνεχής χαρτοπωλείο. Όταν διαδραματίζονται μόνο όταν χρησιμοποιούνται ψηφία ως υπολογισμοί όπως το IBM z / OS και το Parallel Sysplex, ή πλήρως εξελισσόμενοι ρυθμοί απόδοσης.
Εκτεταμένη ολοκληρωμένη εξομοίωση τερματικού πληκτρολογίου / γραφομηχανής, αλλά όχι παράδοση σε γραφικές παραμέτρους χρήσης, αλλά όχι γραφική χρήση μεταξύ των ενεργειακών δαπανών ή τελικών χρηστών για την πραγματοποίηση αυτών των χαρακτηριστικών. Σε πρόσθετες εξειδικευμένες λιγότερο από τις πρώτες σε σχέση με τους λόγους για την εξουσία και την τιμολόγηση των πελατών, το μεγαλύτερο μέρος των προσωπικών υπολογιστών της ευπάθειας NIST, δεδομένου ότι αυτά τα μηχανήματα.

\begin{figure}[H]
	\centering
	\includegraphics[width=0.7\linewidth]{i/18}\
	\caption{Summit}
	\label{fig:18}
\end{figure}


\newpage 

%\appendix
\section{Βιβλιογραφία}
\begin{enumerate}
	\item \url{https://en.wikipedia.org/wiki/ENIAC} \textit{CC-BY-SA-3.0}
	\item \url{https://en.wikipedia.org/wiki/File:ENIAC_Penn1.jpg}
	\item \url{https://en.wikipedia.org/wiki/File:Eniac.jpg}
	\item \url{https://en.wikipedia.org/wiki/File:Two_women_operating_ENIAC.gif}
	\item \url{https://en.wikipedia.org/wiki/File:Street_map_of_Philadelphia_and_surrounding_area.png}
	\item \url{http://www.megaprocessor.com/} James Newman
	\item \url{https://www.youtube.com/watch?v=SbO0tqH8f5I}
	\item \url{https://www.youtube.com/watch?v=LGkkyKZVzug}
	\item \url{https://www.youtube.com/watch?v=ydd6l3iYOZE}
	\item \url{https://www.youtube.com/watch?v=GwHBaSySHmo}
	\item \url{https://en.wikipedia.org/wiki/Graph_embedding}
	\item \url{https://el.wikipedia.org/wiki/TOP500}
	\item \url{https://en.wikipedia.org/wiki/Summit_(supercomputer)}
	\item \url{https://www.olcf.ornl.gov/2018/12/20/ready-for-science-summit-completes-system-acceptance/}
	\item \url{https://en.wikipedia.org/wiki/Mainframe_computer}
\end{enumerate}



\end{document}
